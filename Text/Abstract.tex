\chapter*{Kurzzusammenfassung} % * means not in table of content
\label{chap:Kurzzusammenfassung}

% ca. 150 Worte / Aufgabenstellung, Zielsetzung, verwendete Methoden, Ergebnisse kurz vorstellen, aber nicht diskutieren / Leser entscheidet hier, ob er die Arbeit für lesenswert hält

% deutsch
Für das Projekt \ac{blast} der Hochschulgruppe \ac{hyend} wird eine neue, kompakte und hochleistungsfähige Avionik entwickelt,
die unter extremen Flugbedingungen arbeitet. Die in dieser Arbeit entwickelte Kühlung muss leicht, zuverlässig, wiederverwendbar und für eine
maximale Gehäusetemperatur von $T_\mathrm{C} \leq \SI{89.15}{\celsius}$ für die gesamte Flugdauer ausgelegt sein.
Basierend auf den Anforderungen und Flugbedingungen
wurden drei Konzepte untersucht: reiner Radiator, reines \ac{pcm} und eine hybride Radiator-\ac{pcm}-Lösung. Die Vorauslegung
ergab, dass ein Radiator wegen Aerodynamischer Aufheizung ungeeignet ist. Die hybride Lösung ist möglich, jedoch durch geometrische
Verluste und hohe Luftwärmeströme der Vorauslegung nach mit \SI{4.177}{\kilogram} schwerer als ein einfaches \ac{pcm}
mit \SI{0.347}{\kilogram}. Simulationen der Außenströmung und des \ac{pcm}
bestätigten trotz angenommener Vereinfachungen die Vorauslegungsergebnisse mit einer Masse des hybriden Radiator-\ac{pcm} von \SI{1.625}{\kilogram}.

\chapter*{Abstract} % * means not in table of content
\label{chap:Abstract}
For the \ac{blast} project of the \ac{hyend} university group, a new, compact, and high-performance avionics system is being developed to
operate under demanding flight conditions. The cooling system developed in this work must be lightweight, reliable, reusable, and designed for a maximum
case temperature of $T_\mathrm{C} \leq \SI{89.15}{\celsius}$ for the entire flightduration.
Based on the requirements and flightconditions, three concepts were investigated: pure radiator, pure \ac{pcm}, and a hybrid radiator-\ac{pcm}
solution. Preliminary design showed that a radiator is unsuitable due to aerodynamic heating. The hybrid solution is feasible but, according to
the preliminary design, heavier at \SI{4.177}{\kilogram} due to geometric losses and high convective heat flux than a simple \ac{pcm} at
\SI{0.347}{\kilogram}. Simulations of the external flow and the \ac{pcm} confirmed the preliminary design results despite assumed
simplifications with a mass of the hybrid radiator \ac{pcm} of \SI{1.625}{\kilogram}.