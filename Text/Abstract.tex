\chapter*{Kurzzusammenfassung} % * means not in table of content
\label{chap:Kurzzusammenfassung}

% ca. 150 Worte / Aufgabenstellung, Zielsetzung, verwendete Methoden, Ergebnisse kurz vorstellen, aber nicht diskutieren / Leser entscheidet hier, ob er die Arbeit für lesenswert hält

% deutsch
Für das Projekt \ac{blast} der Hochschulgruppe \ac{hyend} wird eine neue, kompakte und hochleistungsfähige Avionik entwickelt,
die unter extremen Flugbedingungen arbeitet. Die in dieser Arbeit entwickelte Kühlung muss leicht, ausfallsicher und für eine
maximale Sperrschichttemperatur von $T_\mathrm{J} \approx T_\mathrm{C} \leq \SI{85}{\celsius}$ für die gesamte Missionsdauer ausgelegt sein.
Basierend auf den Anforderungen und einer Trajektoriensimulation
wurden drei Konzepte untersucht: reiner Radiator, reines \ac{pcm} und eine hybride Radiator-\ac{pcm}-Lösung. Die Vorauslegung
ergab, dass ein Radiator wegen aerothermaler Aufheizung ungeeignet ist. Die hybride Lösung ist möglich, jedoch durch geometrische
Verluste und hohe Luftwärmeströme der Vorauslegung nach mit \SI{3.835}{\kilogram} schwerer als ein einfaches \ac{pcm}
mit \SI{0.316}{\kilogram}. \ac{cht}-Simulationen der Außenströmung und des \ac{pcm}
bestätigten trotz angenommener Vereinfachungen die Vorauslegungsergebnisse mit einer Masse des hybriden Radiator-\ac{pcm} von \SI{1.522}{\kilogram}.

\chapter*{Abstract} % * means not in table of content
\label{chap:Abstract}
Advanced avionics systems are essential to the success of any experimental rocket.
Ranging from flight computers, telecommunications, and data acquisition to the control
of onboard instrumentation and the rocket itself, high-power microelectronics play a critical role and are often implemented with redundancy.
These systems must be highly compact and capable of withstanding demanding flight conditions, which lead to elevated
power densities that, if not properly managed, can significantly reduce operational lifetime or even cause premature mission failure.
Such is the case in the project \ac{blast} where a novel avionics system is being developed by \ac{hyend} and in need of a thermal management solution.\\

In a first step the demands on the thermal management were set to a maximum junction temperature