\chapter{Zusammenfassung und Ausblick}
\label{chap:Ausblick}
\pagestyle{OnlySection}		% wie ganz oben definiert

Beispielliteraturverweise: 

\begin{enumerate}
	\item Fachzeitschrift
	\item Internetquelle
	\item Buch 
	\item Vorlesungsskript
\end{enumerate}

Anmerkung: Es gibt verschiedene Referenzierungsstile 

\subsection{Flüssig-Gas \ac{pcm}}
Eine weitere Methode zum Thermal-Management, die im Rahmen diese Arbeit nicht analysiert wurde, sind Flüsslig-Gas \ac{pcm}'s, welche generell signifikant höhere latente Wärmen haben,
als die hier analysierten Fest-Flüssig Varianten. Beispielsweise hat Ethanol eine Verdampfungsenthalpie von \SI{918000}{J/kg}, also fast das vierfache von Icosane, bei einer
ähnlichen Dichte. Wegen des großen Volumenanstiegs in die Gasphase von Ethanol, ist ein geschlossenes System, welches extremen Drücken standhalten müsste, eher unhandlich. Hierbei würde
das ablassen vom Ethanol in die Atmosphäre die einzige Möglichkeit sein. Da die Rakete jedoch Ethanol als Treibstoff benötigt ist ein betanken der Kühlung vor dem Start keine logistische Schwierigkeit.\\

Signifikante Probleme mit Ethanol als \ac{pcm} wären der relativ hohe Siedepunkt bei \SI{78}{\degreeCelsius}, der entweder durch Druckregelung auf $< \SI{1}{bar}$ in der oberen Atmosphäre
verringert werrden kann, oder das in Kauf nehmen einer heißer laufenden Avionik. Desweiteren verliert das System den großteil der Thermalen Masse, welche bei unvorhergesehenen Verzögerungen
des Fluges und der Recovery die Avionik schneller überhitzen lassen kann als ein geschlossenes System, in dem auch nach dem Phasenwechsel eine hohe Thermale Masse vorhanden ist.