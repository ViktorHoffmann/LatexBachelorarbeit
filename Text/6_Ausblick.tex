\chapter{Ausblick}
\label{chap:Ausblick}
\pagestyle{OnlySection}		% wie ganz oben definiert

Da diese Arbeit parallel zur Entwicklung der Elektronik entstand, sind vor der Konstruktion und dem Testen des \ac{atm} die exakten
Wärmeströme der Elektronik zu quantifizieren. Genauso müssen nach vorliegen des n-Eicosan dessen thermophysikalische Eigenschaften charakterisiert werden,
da in dieser Arbeit nur öffentlich erhältliche und unabhängige Richtwerte verwendet wurden, die mit großen Toleranzen behaftet sind.
Abgesehen davon ist eine fortführende Entwicklung der wärmetauschenden \ac{pcm}-Struktur notwendig, um die Anforderung der maximalen
Gehäusetemperatur auch zu erfüllen. Alternativ kann durch eine, bereits als notwendig erwähnte, genauere Analyse der Elektronik
die Temperaturanforderung evtl. angepasst werden.

Auch bei der Wärmerohrführung in der finalen Konstruktion muss darauf geachtet werden, dass diese möglichst nah den Bedingungen aus
dem Datenblatt entspricht, da ansonsten eine experimentelle oder numerische Analyse bei ungünstigen Orientierungen sowie mehreren und verteilten Wärmequellen
notwendig ist.