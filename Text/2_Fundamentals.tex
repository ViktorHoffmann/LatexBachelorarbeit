\chapter{Grundlagen}
\label{chap:Grundlagen}			% durch \label{} später auf Element verweisen

In diesem Kapitel werden die Thermodynamischen, Chemischen und Numerischen Grundlagen erläutert, die in dieser Arbeit angewandt wurden. 
% Bücher, aber auch Paper etc. 

\section{Sensible Wärme}\label{sec:sensiblewaerme}

Unter sensibler Wärme versteht man die Eigenschaft von Masse durch eine Temperaturänderung Wärmeenergie zu absorbieren oder abzugeben. Dieses
Phänomen kann durch die änderung der kinetischen Energie von den molekularen Teilchen im System erklärt werden. Durch das einführen von
Wärmeenergie in ein System steigt die kinetische Energie der Teilchen:

\begin{equation}
    c = \frac{\Delta Q}{m \cdot \Delta T}
\end{equation}

Da Elektrionik eine gewisse Eigenmasse hat und meist Teil einer größeren Baugruppe ist, gibt es durch die Sensible Wärme eine Dämpfung
zu Temperaturänderungen, welche jedoch zeitlich von der Wärmeleitfähigkeit der Materialien abhängt.

\section{Latente Wärme}\label{sec:latentewaerme}

Im Gegenteil zur sensiblen Wärme ist latente Wärme, auch Umwandlungsenthalpie genannt, die Eigenschaft von Masse bei einem Phasenwechsel Wärmeenergie
zu absorbieren oder abzugeben, ohne dass dabei die Temperatur sich ändert. Das ist durch die erhöhung der potentiellen Energie der Teilchen,
statt der kinetischen wie bei der sensiblen Wärme, zu verstehen. Effektiv erhöht sich die potentielle Energie durch Änderung der Bindungszustände:

\begin{equation}
    h = \frac{\Delta Q}{m}
\end{equation}

Da die latente Wärme für die meisten Materialien im Fest-Flüssig Übergang um mindestens den Faktor 10 größer ist als die sensible Wärme bei
einem Grad Temperaturerhöhung, kann diese sehr gut zur Absorption von überschüssiger Wärme über längere, jedoch begrenzte Zeiträume verwendet werden,
wobei auch hier die Wärmeleitfähigkeit eine entscheidende Rolle bei der Umsetzung spielt.

\section{Wärmeübertragung}\label{sec:waermeuebertragung}

Um Wärme innerhalb von einem System günstig zu verteilen, oder die Energie aus dem System zu entfernen, gibt es drei Mechanismen.

\subsection{Wärmestrahlung}\label{sec:strahlung}

Bei der Wärmestrahlung geben Teilchen beim aufnehmen oder abgeben kinetischer Energie eine Gewisse Menge an Energie in Form
von Elektromagnetischer Strahlung ab. Da die Strahlungsleistung von der vierten Potenz der Temperatur
abhängt, ist dieser Modus erst bei sehr hohen Temperaturen dimensionierend, kann jedoch im Vakuum dominant sein:

\begin{equation}
    \label{eq:radiation}
    \dot{Q}=\sigma\epsilon A T^{4}
\end{equation}

\subsection{Wärmeleitung}\label{sec:waermeleitung}

Bei der Wärmeleitung wird Wärmeenergie in einem Körper durch diffusion der kinetischen Energie der Teilchen verteilt.
Die Wärmestromdichte in einem Temperaturgradienten wird durch das Fourier-Gesetz beschrieben:

\begin{equation}
    \vec{\dot{q}} = -\lambda \nabla T
\end{equation}

%todo

\subsection{Konvektion}\label{sec:konvektion}

Bei der Konvektion wird Wärmeenergie durch Massenaustausch transportiert. Bei der erzwungenen Konvektion bekommt das Fluid durch äußere Kräfte
eine relative Geschwindigkeit, die zum Massenaustausch führt. Andererseits resultiert bei der natürlichen Konvektion nur die eigene
inhomogene Temperaturverteilung, durch beispielsweise eine anliegende heiße Wand, zu einem Temperaturanstieg und infolge dessen zu einem
Dichteanstieg, der in einem Beschleunigungsfeld zu Auftriebskräften und automatischer Bewegung des Fluids führt.
Für den Wärmeübergang zwischen Fluid und Festkörper ergibt sich:

\begin{equation}
    \dot{Q}=\alpha A \Delta T 
\end{equation}

Für den spezifischen Wärmestrom zwischen Fluid und Festkörper folgt daraus:

\begin{equation}
  \label{eq:qdot}
  \dot{q} = \alpha \ (T_r - T_w)
\end{equation}

Der Wärmeübergangskoeffizient $\alpha$ wird aus der Nußelt Beziehung genommen. Diese lautet
für laminare Grenzschichten im Gültigkeitsbereich $\text{Re} < \text{Re}_k\\\left(\text{Re}_k \approx 5 \cdot 10^5\right)$ und $0,6 \leq \text{Pr} \leq 2000$:

\begin{equation}
  \label{eq:nusselt_laminar}
  \text{Nu}_x = \frac{\alpha_x x}{\lambda} = 0,332 \ \text{Pr}^{\frac{1}{3}} \ \text{Re}_x^{\frac{1}{2}}
\end{equation}

für turbulente Grenzschichten mit Gültigkeitsbereich: $5 \cdot 10^7 \leq \text{Re}_L \leq 10^7$ und $ 0,6 \leq \text{Pr} \leq 2000$ lautet die Gleichung:

\begin{equation}
  \label{eq:nusselt_turbulent}
  \text{Nu}_x = \frac{\alpha_x x}{\lambda} = 0,0296 \ \text{Re}_x^{0,8} \ \text{Pr}^{\frac{1}{3}}
\end{equation}

Für die Reynoldszahl und Prandtlzahl werden die folgenden zwei Gleichungen verwendet:
\newline
\noindent\begin{minipage}{.5\linewidth}
  \begin{equation}
    \label{eq:reynolds}
    \text{Re}_x = \frac{V \rho x}{\eta}
  \end{equation}
\end{minipage}%
\begin{minipage}{.5\linewidth}
  \begin{equation}
    \label{eq:prandtl}
    \text{Pr} = \frac{c_p \eta}{\lambda}
  \end{equation}
\end{minipage}

Die Dynamische Viskosität wird mittels der Sutherlands-Formel~\ref{eq:dynamische_viskositaet} berechnet, und die Recoverytemperatur mittels der adiabaten Strömungsgleichung~\ref{eq:recovery_temperatur}.
In diesem Fall wird die Recoverytemperatur statt der Freistromtemperatur benötigt, da die signifikanten Wärmestrome weit über Mach 0.3 erwartet werden und somit die aerodynamische
Aufheizung berücksichtigt wird.

\begin{equation}
  \label{eq:dynamische_viskositaet}
  \eta = \eta_0 \frac{T_0 + C}{T_{\infty} + C} {\left( \frac{T_{\infty}}{T_0} \right)}^{\frac{2}{3}}
\end{equation}

\begin{equation}
  \label{eq:recovery_temperatur}
  T_r = T_{\infty} \left( 1 + r \frac{\kappa + 1}{2} \text{Ma}^2 \right)
\end{equation}

Der Recovery-Faktor kann mittels der folgenden Gleichung approximiert werden:

\begin{equation}
  \label{eq:recovery_faktor}
  r = \frac{2}{\left(\kappa - 1\right) \text{Ma}_e^2} \left(\frac{T_{a_{w}}}{T_e} - 1\right) \approx
  \begin{cases}
    \sqrt[3]{\text{Pr}} & \text{für turbulente Grenzschicht}\\
    \sqrt{\text{Pr}} & \text{für laminare Grenzschicht}
  \end{cases}
\end{equation}

\subsection{Simulation}

