\chapter{Grundlagen}
\label{chap:Grundlagen}			% durch \label{} später auf Element verweisen

In diesem Kapitel werden die Thermodynamischen, Chemischen und Numerischen Grundlagen erläutert, die in dieser Arbeit angewandt wurden. 
% Bücher, aber auch Paper etc. 

\section{Sensible Wärme}\label{sec:sensiblewaerme}

Unter sensibler Wärme versteht man die Eigenschaft von Masse durch eine Temperaturänderung Wärmeenergie zu absorbieren oder abzugeben. Dieses
Phänomen kann durch die änderung der kinetischen Energie von den molekularen Teilchen im System erklärt werden. Durch das einführen von
Wärmeenergie in ein System steigt die kinetische Energie der Teilchen:

\begin{equation}
    c = \frac{\Delta Q}{m \cdot \Delta T}
\end{equation}

$c$ beschreibt die spezifische Wärmekapazität, welche entweder bei konstantem Druck oder konstantem Volumen angegeben ist,
$Q$ ist die Wärmeenergie, $m$ die Masse und $T$ die Temperatur

Da Elektrionik eine gewisse Eigenmasse hat und meist Teil einer größeren Baugruppe ist, gibt es durch die Sensible Wärme eine Dämpfung
zu Temperaturänderungen, welche jedoch zeitlich von der Wärmeleitfähigkeit der Materialien abhängt.

\section{Latente Wärme}\label{sec:latentewaerme}

Im Gegenteil zur sensiblen Wärme ist latente Wärme, auch Umwandlungsenthalpie genannt, die Eigenschaft von Masse bei einem Phasenwechsel Wärmeenergie
zu absorbieren oder abzugeben, ohne dass dabei die Temperatur sich ändert. Das ist durch die erhöhung der potentiellen Energie der Teilchen,
statt der kinetischen wie bei der sensiblen Wärme, zu verstehen. Effektiv erhöht sich die potentielle Energie durch Änderung der Bindungszustände.
Die Stoffkonstante der Umwandlungsenthalpie ist die spezifische Umwandlungsenthalpie $h$:

\begin{equation}
    h = \frac{\Delta Q}{m}
\end{equation}

Zu beachten ist, dass die Konvention der Schreibweise für die massenspezifische Fest-Flüssig Umwandlungsenthalpie spezifische Schmelzenthalpie
ist, aber für die massenspezifische Flüssig-Gas Umwandlungsenthalpie nur Verdampfungsenthalpie ist.\\

Die latente Wärme ist für die meisten Materialien im Fest-Flüssig Übergang um mindestens den Faktor 10 größer als die sensible Wärme bei
einem Grad Temperaturerhöhung. Genauso ist die Verdampfungsenthalpie vom Flüssig-Gas Übergang meist um etwa den Faktor 10 größer als die 
spezifische Schmelzenthalpie~\cite{fusion-vaporization}.

\section{Wärmeübertragung}\label{sec:waermeuebertragung}

Um Wärme innerhalb von einem System günstig zu verteilen, oder die Energie aus dem System zu entfernen, gibt es drei Mechanismen.

\subsection{Wärmestrahlung}\label{sec:strahlung}

Bei der Wärmestrahlung geben Teilchen beim aufnehmen oder abgeben kinetischer Energie eine Gewisse Menge an Energie in Form
von Elektromagnetischer Strahlung ab. Da die Strahlungsleistung von der vierten Potenz der Temperatur
abhängt, ist dieser Modus erst bei sehr hohen Temperaturen dimensionierend, kann jedoch im Vakuum dominant sein:

\begin{equation}
    \label{eq:radiation}
    \dot{Q}=\sigma\epsilon A T^{4}
\end{equation}

$\dot{Q}$ ist der Wärmestrom, $\sigma$ die Stefan-Bolzmann-konstante, $\epsilon$ der Emissionsgrad, welcher von der Wellenlänge
abhängt, $A$ die Fläche und $T$ die Temperatur des Radiators.

\subsection{Wärmeleitung}\label{sec:waermeleitung}

Bei der Wärmeleitung wird Wärmeenergie in einem Körper durch diffusion der kinetischen Energie der Teilchen verteilt.
Die Wärmestromdichte $\dot{q}$ in einem Temperaturgradienten wird durch das Fourier-Gesetz beschrieben:

\begin{equation}
    \vec{\dot{q}} = -\lambda \nabla T
\end{equation}

Hier ist $\lambda$ die Wärmeleitfähigkeit des Materials.
Für eine eindimensionale Wand ergibt sich die Gleichung zu:

\begin{equation}
  hilfe
\end{equation}

\subsection{Konvektion}\label{sec:konvektion}

Bei der Konvektion wird Wärmeenergie durch Massenaustausch transportiert. Bei der erzwungenen Konvektion bekommt das Fluid durch äußere Kräfte
eine relative Geschwindigkeit, die zum Massenaustausch führt. Andererseits resultiert bei der natürlichen Konvektion nur die eigene
inhomogene Temperaturverteilung, durch beispielsweise eine anliegende heiße Wand, zu einem Temperaturanstieg und infolge dessen zu einem
Dichteanstieg, der in einem Beschleunigungsfeld zu Auftriebskräften und automatischer Bewegung des Fluids führt.
Für den Wärmeübergang zwischen Fluid und Festkörper ergibt sich:

\begin{equation}
    \dot{Q}=\alpha A \Delta T 
\end{equation}

Für den spezifischen Wärmestrom zwischen Fluid und Festkörper folgt daraus:

\begin{equation}
  \label{eq:qdot_freestream}
  \dot{q} = \alpha \ (T_f - T_w)
\end{equation}

Der Wärmeübergangskoeffizient $\alpha$ wird aus der Nußelt Beziehung genommen. Diese lautet
für laminare Grenzschichten im Gültigkeitsbereich $\text{Re} < \text{Re}_k\\\left(\text{Re}_k \approx 5 \cdot 10^5\right)$ und $0,6 \leq \text{Pr} \leq 2000$:

\begin{equation}
  \label{eq:nusselt_laminar}
  \text{Nu}_x = \frac{\alpha_x x}{\lambda} = 0,332 \ \text{Pr}^{\frac{1}{3}} \ \text{Re}_x^{\frac{1}{2}}
\end{equation}

für turbulente Grenzschichten mit Gültigkeitsbereich: $5 \cdot 10^7 \leq \text{Re}_L \leq 10^7$ und $ 0,6 \leq \text{Pr} \leq 2000$ lautet die Gleichung:

\begin{equation}
  \label{eq:nusselt_turbulent}
  \text{Nu}_x = \frac{\alpha_x x}{\lambda} = 0,0296 \ \text{Re}_x^{0,8} \ \text{Pr}^{\frac{1}{3}}
\end{equation}

Für die Reynoldszahl und Prandtlzahl werden die folgenden zwei Gleichungen verwendet:
\newline
\noindent\begin{minipage}{.5\linewidth}
  \begin{equation}
    \label{eq:reynolds}
    \text{Re}_x = \frac{V \rho x}{\eta}
  \end{equation}
\end{minipage}%
\begin{minipage}{.5\linewidth}
  \begin{equation}
    \label{eq:prandtl}
    \text{Pr} = \frac{c_p \eta}{\lambda}
  \end{equation}
\end{minipage}

Die Dynamische Viskosität wird mittels der Sutherlands-Formel~\ref{eq:dynamische_viskositaet} berechnet, und die Recoverytemperatur mittels der adiabaten Strömungsgleichung~\ref{eq:recovery_temperatur}.
In diesem Fall wird die Recoverytemperatur statt der Freistromtemperatur benötigt, da die signifikanten Wärmestrome weit über Mach 0.3 erwartet werden und somit die aerodynamische
Aufheizung durch adiabatische Kompression berücksichtigt wird. Dementsprechend wird $T_\text{f}$ aus \ref{eq:qdot_freestream} zu $T_\text{r}$:

\begin{equation}
  \label{eq:qdot_recovery}
  \dot{q} = \alpha \ (T_\text{r} - T_\text{w})
\end{equation}

\begin{equation}
  \label{eq:dynamische_viskositaet}
  \eta = \eta_0 \frac{T_0 + C}{T_{\infty} + C} {\left( \frac{T_{\infty}}{T_0} \right)}^{\frac{2}{3}}
\end{equation}

Eine Strömung ist im Überschallbereich, wenn ihre Machzahl größer als 1 ist:

\begin{equation}
  \label{eq:machzahl}
  Ma = \frac{U}{a}
\end{equation}

Wobei $U$ die Strömungsgeschwindigkeit und $a$ die lokale Schallgeschwindigkeit ist. Im Überschallbereich treten verschiedene
Effekte durch die Kompressibilität der Strömung auf, wie große Unterschiede in Dichte, Temperatur und Druck.

Äußern tun sich diese Effekte durch Stoßwellen mit sprunghaftem Anstieg der Größen, Expansionsfächer mit Abfall der Größen und
Temperaturerhöhungen durch die adiabate Kompression der Luft:

\begin{equation}
  \label{eq:recovery_temperatur}
  T_r = T_{\infty} \left( 1 + r \frac{\kappa + 1}{2} \text{Ma}^2 \right)
\end{equation}

Der Recovery-Faktor kann mittels der folgenden Gleichung berechnet werden:

\begin{equation}
  \label{eq:recovery_faktor}
  r = \frac{2}{\left(\kappa - 1\right) \text{Ma}_e^2} \left(\frac{T_{a_{w}}}{T_e} - 1\right) \approx
  \begin{cases}
    \sqrt[3]{\text{Pr}} & \text{für turbulente Grenzschicht}\\
    \sqrt{\text{Pr}} & \text{für laminare Grenzschicht}
  \end{cases}
\end{equation}

\subsection{Simulation}

Die numerische Ströumgssimulation (\ac{cfd}) ist ein Verfahren zur Berechnung von Strömungs- und Wärmeübergangsprozessen
mithilfe numerischer Methoden. \ac{cfd} erlaubt die Untersuchung komplexer Geometrien und Betriebsbedingungen,
die experimentell nur schwer oder gar nicht möglich sind. Ziel ist es, die Navier-Stokes-Gleichungen in differentieler Form auf einer
diskreten Gitterstruktur zu lösen. Diese Erhaltungsgleichungen sind die Massenerhaltung:

\begin{equation}
  \label{eq:navier_massenerhaltung}
  \frac{\delta \rho}{\delta t} + \nabla \left( \rho \vec{u} \right) = 0
\end{equation}

Impulserhaltung:

\begin{equation}
  \label{eq:navier_impulserhaltung}
  \frac{\delta \left( \rho \vec{u} \right)}{\delta t} + \nabla \left( \rho \vec{u} \vec{u} \right) = -\nabla p + \nabla \tau + \rho \vec{g}
\end{equation}

und Energieerhaltung:

\begin{equation}
  \label{eq:navier_energieerhaltung}
  \frac{\delta \left(\rho \vec{u}\right)}{\delta t} + \nabla \left[\left(\rho E + p\right)\vec{u}\right] = \nabla \left(k \nabla T\right) + \Phi
\end{equation}

Hier sind $\rho$ die Dichte, $\vec{u}$ der Geschwindigkeitsvektor, $p$ der statische Druck, $\tau$ der Spannungstensor,
$\vec{g}$ die Gravitationsbeschleunigung, $E$ die spezifische Gesamtenergie, $T$ die Temperatur, $k$ die Wärmeleitfähigkeit
und $\Phi$ der viskose Dissipationsterm.

In der verwendeten Software ANSYS Fluent werden diese Gleichungen über die Finite-Volumen-Methode gelöst. Dabei werden die Erhaltungsgleichungen
über diskrete Kontrollvolumina integriert, wordurch für jede Zelle ein algebraisches Gleichungssystem entsteht. Dieses
wird iterativ gelöst, bis vorgegebene Konvergenzkriterien erfüllt sind.

Speziell für \ac{pcm} kann die Bossinesq-Approximation 