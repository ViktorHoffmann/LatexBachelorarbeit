 \chapter{Grundlagen}
\label{chap:Grundlagen}			% durch \label{} später auf Element verweisen

In diesem Kapitel werden die Thermodynamischen, Chemischen und Numerischen Grundlagen erläutert, die in dieser Arbeit angewandt wurden. 
% Bücher, aber auch Paper etc. 

\section{Sensible Wärme}\label{sec:sensiblewaerme}

Unter sensibler Wärme versteht man die Eigenschaft von Masse Energie zu absorbieren, oder abzugeben und dadurch die Temperatur zu ändern. Dieses
Phänomen kann durch die Änderung der kinetischen Energie von den molekularen Teilchen im System erklärt werden. Durch das einführen von Energie in
ein System steigt die kinetische Energie der Teilchen, welche diese Energie über molekurare Interaktionen verteilen:

\begin{equation}
    c = \frac{\Delta Q}{m \cdot \Delta T}
\end{equation}

Da Elektrionik eine gewisse Eigenmasse hat und meist Teil einer größeren Baugruppe ist, gibt es durch die Sensible Wärme eine Dämpfung
zu Temperaturänderungen, welche jedoch zeitlich von der Wärmeleitfähigkeit der Materialien abhängt.

\section{Latente Wärme}\label{sec:latentewaerme}

Im kontrast zur sensiblen Wärme ist latente Wärme, auch Umwandlungsenthalpie genannt, die Eigenschaft von Masse bei einem Phasenwechsel Energie
zu absorbieren oder abzugeben, ohne dass dabei die Temperatur sich ändert. Das ist durch die erhöhung der potentiellen Energie der Teilchen,
statt der kinetischen wie bei der sensiblen Wärme, zu verstehen. Effektiv erhöht sich also nicht die Geschwindigkeit der Teilchen sondern die
Anzahl an Stellen an die die Teilchen sich bewegen können.

\begin{equation}
    h = \frac{\Delta Q}{m}
\end{equation}

Da die latente Wärme für die meisten Materialien im Fest-Flüssig Übergang um mindestens den Faktor 10 größer ist als die sensible Wärme bei
einem Grad Temperaturerhöhung, kann diese sehr gut zur Absorption von überschüssiger Wärme über längere, aber finite, Zeiträume verwender werden,
wobei auch hier die Wärmeleitfähigkeit eine entscheidende Rolle bei der Umsetzung spielt.

\section{Wärmestrahlung}\label{sec:radiator}
Die Wärmestrahlung ist einer von drei Wegen, wie sich Wärme in einem System verteilt.

\begin{equation}
    Q=\sigma\epsilon A T^{4}
\end{equation}

\section{Hybrid Lösung}
\label{sec:hybridloesung}

\newpage
