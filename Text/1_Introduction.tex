\chapter{Einführung}			
\label{sec:Introduction}
\pagenumbering{arabic}		%ab hier: arabische Seitenzahlen (Hauptteil)

% Problemstellung und Lösungsansätze, 2-3 Seiten / keine Ergebnisse & Folgerungen /

Für die im Rahmen des aktuellen Projekts, \ac{blast} der Studentischen Hochschulgruppe \ac{hyend}, neu entwickelte Avionik soll eine Therman-Management-Lösung entwickelt werden.

\section{Darstellung des Problems}

-Neues Projekt mit eigener Avionik\\
-Leistungsstarke Avionik mit Redundantem \ac{fcc}\\
-Schwierige Umweltbedingungen\\
-(Pad ist nicht teil des Problems)
\newline
Beim Projekt \ac{blast} der studentischen Hochschulgruppe \ac{hyend} wird eine neue Avionik mit einem selbst entwickelten \ac{fcc} gebaut. Durch 

\section{Zielsetzung der Arbeit}

-Entwicklung eines Thermal-Managements für die komplette Flugdauer\\
-Ausfallsicher\\
-Leichtbau\\
-Wiederverwendbar

\section{Lösungsweg}

-vorauslegung\\
-simulation\\
\cite{Claudio-2018, Ho-2021, Isaacs-2017, Abdel-2024, Xu-2022, Pavia-2015, Yang-2015, ST-guide, NASA-2023, Gilmore-2002, Hume-2022}\\
$\mathrm{T_c}$ soll auf \SI{85}{\celsius} bleiben -> Beispielrechnung der Ausfallwahrscheinlichkeit für STM32?\\
Mit trajektorien simulation und längs angeströmter turbulenter Platte bekomme ich spezifischen Wärmestrom an fixer Stelle über dein Flug\\
Aus flugmaxx krieg ich dauer und stärke der beschleunigung -> Ansys, transient\\
Rest des Fluges ist Mikrogravitation -> Ansys, transient\\