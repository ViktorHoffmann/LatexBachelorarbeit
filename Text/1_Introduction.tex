\chapter{Einführung}			
\label{sec:Introduction}
\pagenumbering{arabic}		%ab hier: arabische Seitenzahlen (Hauptteil)

% Problemstellung und Lösungsansätze, 2-3 Seiten / keine Ergebnisse & Folgerungen /

Eine leistungsstarke Avionik ist ein Grundstein jeder erfolgreichen Experimentalrakete.
Ob es hierbei um Flugcomputer, Telekommunikation, Datenerfassung oder auch aktive Steuerung und Regelung von
Instrumenten und dem Fahrzeug während des Flugs geht, Hochleistungsmikroelektronik ist immer gefragt und muss oft redundant ausgeführt sein.
Diese Elektronik, die zudem noch extrem kompakt sein muss und extremen Bedingungen ausgesetzt wird, kommt jedoch mit einer
substanziellen Wärmeleistung und Wärmestromdichte die, bei mangelhafter Rücksicht zu reduzierter Lebensdauer der Avionik führen
kann oder sogar die Mission frühzeitig scheitern lässt.
Diese Arbeit befasst sich mit der lösung des dargestellten problems für das Projekt \ac{blast} der studentischen Hochschulgruppe \ac{hyend}
wo eine neue avionik entwickelt wird und eine Thermal-Management-Lösung (Kühlung) benötigt.\\

\section{Darstellung des Problems}

-Neues Projekt mit eigener Avionik\\
-Leistungsstarke Avionik mit Redundantem \ac{fcc}\\
-Schwierige Umweltbedingungen\\
-(Pad ist nicht teil des Problems)
\newline
Beim Projekt \ac{blast} der studentischen Hochschulgruppe \ac{hyend} wird eine neue Avionik mit einem selbst entwickelten \ac{fcc} gebaut. Durch 

\section{Zielsetzung der Arbeit}

-Entwicklung eines Thermal-Managements für die komplette Flugdauer\\
-Ausfallsicher\\
-Leichtbau\\
-Wiederverwendbar

\section{Lösungsweg}

-vorauslegung\\
-simulation\\
\cite{Claudio-2018, Ho-2021, Isaacs-2017, Abdel-2024, Xu-2022, Pavia-2015, Yang-2015, ST-guide, NASA-2023, Gilmore-2002, Hume-2022}\\
$\mathrm{T_c}$ soll auf \SI{85}{\celsius} bleiben -> Beispielrechnung der Ausfallwahrscheinlichkeit für STM32?\\
Mit trajektorien simulation und längs angeströmter turbulenter Platte bekomme ich spezifischen Wärmestrom an fixer Stelle über dein Flug\\
Aus flugmaxx krieg ich dauer und stärke der beschleunigung -> Ansys, transient\\
Rest des Fluges ist Mikrogravitation -> Ansys, transient\\